\documentclass[letterpaper,11pt]{article}

\usepackage{latexsym}
\usepackage[empty]{fullpage}
\usepackage{titlesec}
\usepackage{marvosym}
\usepackage[usenames,dvipsnames]{color}
\usepackage{verbatim}
\usepackage{enumitem}
\usepackage{xifthen}
% \usepackage[pdftex]{hyperref}
\usepackage{fancyhdr}
\usepackage{fontawesome5}
\usepackage{multirow}
\usepackage{multicol}
\usepackage{natbib}
\usepackage{bibentry}
\bibliographystyle{apalike}


\usepackage{my_macros}

\definecolor{xlinkcolor}{cmyk}{1,1,0,0}
\usepackage[
  colorlinks=true,
  linkcolor=xlinkcolor,
  citecolor=xlinkcolor,
  filecolor=xlinkcolor,
  urlcolor=xlinkcolor,
]{hyperref}

\pagestyle{fancy}
\fancyhf{} % clear all header and footer fields
\fancyfoot{}
% Ensure footers reserve space to avoid fancyhdr warning
\setlength{\footskip}{6pt}
\renewcommand{\headrulewidth}{0pt}
\renewcommand{\footrulewidth}{0pt}

% Adjust margins
\addtolength{\oddsidemargin}{-0.375in}
\addtolength{\evensidemargin}{-0.375in}
\addtolength{\textwidth}{1in}
\addtolength{\topmargin}{-.5in}
\addtolength{\textheight}{1.0in}

\urlstyle{same}

\raggedbottom
\raggedright
\setlength{\tabcolsep}{0in}

% Sections formatting
\titleformat{\section}{
  \vspace{-4pt}\scshape\raggedright\large
}{}{0em}{}[\color{black}\titlerule \vspace{-5pt}]

%-------------------------
% Custom commands
\newlength{\mylen}
\setbox1=\hbox{$\bullet$}\setbox2=\hbox{\tiny$\bullet$}
\setlength{\mylen}{\dimexpr0.5\ht1-0.5\ht2}
\renewcommand\labelitemi{\raisebox{\mylen}{\small$\bullet$}}

\newcommand{\resumeItem}[2]{
  \item\small{
    \textbf{#1}{: #2 \vspace{-2pt}}
  }
}

\newcommand{\smallHeading}[1]{\small{\textbf{#1}:\vspace{-5pt}}}

\newcommand{\resumeSubheading}[4]{
  \vspace{-1pt}\item
    \begin{tabular*}{0.97\textwidth}{l@{\extracolsep{\fill}}r}
      \textbf{#1} & #2 \\
      \textit{\small#3} & \textit{\small #4} \\
    \end{tabular*}\vspace{-7pt}
}

\newcommand{\resumeSubItem}[2]{\resumeItem{#1}{#2}\vspace{-4pt}}

\renewcommand{\labelitemii}{$\circ$}

\newcommand{\resumeSubHeadingListStart}{\begin{itemize}[leftmargin=*]}
\newcommand{\resumeSubHeadingListEnd}{\end{itemize}}
\newcommand{\resumeItemListStart}{\begin{itemize}}
\newcommand{\resumeItemListEnd}{\end{itemize}\vspace{-5pt}}

\newcommand{\paperEntry}[5]{
  \item #1, ..., \textbf{Burrow, Anthony}, et al. (#2). \textit{#3}. {#4}
  \ifthenelse{\isempty{#5}}{}{, doi: \href{https://doi.org/#5}{#5}}
}

\newcommand{\paperEntryFirstAuthor}[5]{
  \item \textbf{Burrow, Anthony}, #1, et al. (#2). \textit{#3}. #4
  \ifthenelse{\isempty{#5}}{}{, doi: \href{https://doi.org/#5}{#5}}
}

\newcommand{\paperEntryShort}[4]{
  \href{https://doi.org/#1}{\textbf{Burrow, Anthony}, et al. (#2)}.
  \textit{#3}. #4
}

% \newcommand{\educationEntry}[5]{
%   \vspace{-1pt}\item
%     \begin{tabular*}{0.97\textwidth}{l@{\extracolsep{\fill}}r}
%       \textbf{#1}\ifthenelse{\isempty{#2}}{}{; GPA: #2}  & #3 \\
%       \textit{\small#4} & \textit{\small #5} \\
%     \end{tabular*}\vspace{-5pt}
% }

\newcommand{\educationEntry}[5]{
  \vspace{-1pt}
  \item
    \begin{tabular*}{0.97\textwidth}[t]{l}
      \textbf{#1}\ifthenelse{\isempty{#4}}{}{ (#4)} \\
      #3, \textit{\small#2}
      \ifthenelse{\isempty{#5}}{}{\\ #5 \\}
    \end{tabular*}
  \vspace{-5pt}
}

% \newcommand{\emphasize}[1]{\textbf{#1}}
\newcommand{\emphasize}[1]{#1}

\newcommand{\technology}[2]{
  \small
  \textbf{#1}: #2
  \vspace{3pt}
}

% -------------------------------------------

\begin{document}

\nobibliography{mybib}


% Header
\begin{tabular*}{\textwidth}{l@{\extracolsep{\fill}}l}
  \multirow{3}{*}{\textbf{\huge Anthony Burrow, Ph.D.}}
  & \faIcon{envelope} \enspace anthony.r.burrow@gmail.com \\
  & \faIcon{linkedin} \enspace \href{https://www.linkedin.com/in/anthony-burrow}{linkedin.com/in/anthony-burrow} \\
  & \faIcon{home} \enspace \href{https://anthonyburrow.github.io}{anthonyburrow.github.io} \\
  % & \faGithub \enspace \href{http://www.github.com/anthonyburrow}{github.com/anthonyburrow} \\
\end{tabular*}


\vspace{4pt}


%
% Objective
%


\section{Objective}

Data scientist with a Ph.D. in Physics and experience leveraging machine
learning to model complex data sets in astrophysics. Seeking to apply my
advanced skill set in model production, data visualization, and software
development in Python and C++ to deliver insightful data-driven solutions to
business goals.

\vspace{5pt}


%
% Education | Certifications
%


\begin{minipage}[t]{0.48\textwidth}
\section{Education}

\resumeSubHeadingListStart

  % \vspace{-8pt}
  \setlength\itemsep{2pt}
  \educationEntry
    {Ph.D. in Physics}
    {University of Oklahoma}{July 2024}{GPA: 3.92; \href{https://www.parchment.com/u/award/5ebe324637723257bda3659f8c5bb35c}{Verification}}
    {\href{https://shareok.org/handle/11244/340468}{View Dissertation}}

  \educationEntry
    {B.S. in Astrophysics}
    {University of Oklahoma}{May 2017}{GPA: 3.91}
    {}

\resumeSubHeadingListEnd

\end{minipage}%
\hspace{0.04\textwidth}%
\begin{minipage}[t]{0.48\textwidth}
\section{Certifications}

\resumeSubHeadingListStart

  \item
      \textbf{IBM Professional Certificate}
      \begin{itemize}[leftmargin=*, topsep=-2pt]
        \item Data Science, Aug. 2024 (\href{https://www.coursera.org/account/accomplishments/specialization/certificate/1CX1HP3UEHJ1}{View})
        \item[] Enhancing skills in the data science lifecycle --- ETL, data
          analysis, data visualization, SQL, machine learning, etc. --- to
          derive insights from data.
        % \item DevOps and Software Engineering, \textit{Ongoing}
      \end{itemize}

\resumeSubHeadingListEnd

\end{minipage}

\vspace{-3pt}


%
% Relevant Experience
%


\section{Relevant Experience}
\resumeSubHeadingListStart
  \setlength\itemsep{2pt}

  \normalsize
  \resumeSubheading
    {Generative AI Trainer}{Dec. 2024 -- Present}
    {Independent Contractor}{}

    \begin{itemize}
      \small
      \setlength\itemsep{0pt}
      \item Serve as an expert in physics, mathematics, and programming to
        train and evaluate responses from large language models (LLMs) to
        STEM-based prompts.
      \item Assess the validity of evaluations from other AI trainers to
        maintain quality assurance in model training data.
    \end{itemize}
  \vspace{-5pt}

  \normalsize
  \resumeSubheading
    {Research Assistant (Data Science)}{July 2019 -- July 2024}
    {University of Oklahoma}{Norman, OK}

    \begin{itemize}
      \small
      \setlength\itemsep{0pt}
      \item Employed machine learning algorithms to establish a new, optimized
        standard of classifying astronomical objects.
      \item Designed an innovative method of extrapolating and predicting
        astronomical data by leveraging dimensionality reduction techniques,
        with extrapolated results matching observed data within 2--5\%.
      \item Led collaborative statistical analyses with leading researchers
        from around the world, resulting in two published reports to a
        peer-reviewed journal that interpret findings using data visualizations
        and highly technical writing.
      \item Developed and maintained open-source software to enhance
        accessibility of models for the scientific community.
      \item Utilize hybrid cloud computing infrastructure to synthesize physics
        models for comparative data analysis.
    \end{itemize}
    \vspace{-7pt}
    \hspace*{1.1em}\smallHeading{Products}
    \begin{itemize}
      \small
      \setlength\itemsep{0pt}
      \item \href{https://anthonyburrow.github.io/projects/WDClassification}{White Dwarf Classifier}:
        Developed several ML models (e.g., random forest, gradient boosting,
        SVM) to classify several groups of objects within over 90\% accuracy,
        achieving results that improve upon previously published work.
      \item \href{https://anthonyburrow.github.io/projects/SNEx}{SNEx} (Python):
        Spectrum extrapolation into near-infrared wavelengths using
        principal component analysis.
      \item \href{https://anthonyburrow.github.io/projects/Spextractor}{Spextractor} (Python):
        Fast spectrum-smoothing by implementing interpolation via Gaussian
        process regression.
      \item See Publications section (next page).
      % \item \href{https://anthonyburrow.github.io/projects/SNIaDCA}{SNIaDCA} (Python):
      %   Wrapper for probablistically classifying supernovae with
      %   Gaussian mixture models.
    \end{itemize}
  \vspace{-5pt}

  \normalsize
  \resumeSubheading
    {Undergraduate Research Assistant (Data Analysis)}{June 2015 -- May 2017}
    {University of Oklahoma}{Norman, OK}

    \begin{itemize}
      \small
      \setlength\itemsep{0pt}
      \item Created Python scripts needed for data analysis, discovering
        several stars with time-variable properties.
      \item Calibrated and processed raw image data by removing multiple
        sources of noise and bias.
      \item Conducted multiple remote observations at the Apache Point
        Observatory to obtain more raw data for analysis.
    \end{itemize}

\resumeSubHeadingListEnd

\vspace{-10pt}


%
% Technical Skills
%


\section{Technical Skills}

\technology{Programming \& Scripting}
{Python, C/C++, SQL, R, C\#, Slurm, Make, CMake, Fortran, \LaTeX}

\technology{Python Libraries}
{NumPy, pandas, scikit-learn, Tensorflow, matplotlib, SciPy, Astropy, pytest}

\technology{Development Tools}
{Docker, Power BI, JupyterLab, Visual Studio, RStudio}

\technology{Platforms}
{Linux/UNIX (Ubuntu, Arch), Windows}

\technology{Version Control}
{Git, GitHub, GitHub Actions workflows, Continuous Integration \& Deployment
(CI/CD)}

% \technology{Data Skills}
% {Machine Learning, Data Wrangling, Data Analysis, Data Visualization,
% Statistics, Model Evaluation, Regression, Classification, Cluster Analysis,
% Feature Engineering, Parameter Optimization}

% {Data Wrangling, Data Analysis,
% Statistics, Model Evaluation, Regression, Classification, Cluster Analysis,
% Feature Engineering, Parameter Optimization}

\vspace{-3pt}


%
% Relevant Projects
%


\section{Additional Software Projects}
\resumeSubHeadingListStart

  % \item \href{https://anthonyburrow.github.io/projects/WDClassification}{White Dwarf Classifier}:
  %   Developed several classification models (e.g., random forest, gradient
  %   boosting, SVM) to classify astronomical data within up to 98\%, achieving
  %   results that improve upon previously published work.

  \item \href{https://github.com/anthonyburrow/SpecFit}{SpecFit} (Python, C++):
    Efficient spectrum-fitting of various integrated models using non-linear
    optimization, utilizing \texttt{pybind11} to allow fast C++ functionality
    within Python.

  \item \href{https://anthonyburrow.github.io/projects/Rad1D}{Rad1D}:
    Radiative transfer code that utilizes iterative matrix operations
    to solve complex equations that explain light passing through a medium.
    Written in C++ with a Python wrapper, implementing unit testing for
    seamless integration.

  \item \href{https://anthonyburrow.github.io/projects/Hydro1D}{Hydro1D}:
    Hydrocode that simulates the time-dependent shock dynamics of stellar
    material (C++, Python).

\resumeSubHeadingListEnd


%
% Publications
%


\section{Publications}
\resumeSubHeadingListStart

  \paperEntryFirstAuthor{Baron, E., Burns, Christopher R.}{2024}
    {Extrapolation of Type Ia Supernova Spectra into the Near-Infrared Using PCA}
    {\apj}{10.3847/1538-4357/ad3c45}
  \paperEntryFirstAuthor{Baron, E., Ashall, C.}{2020}
    {Carnegie Supernova Project: Classification of Type Ia Supernovae}
    {\apj}{10.3847/1538-4357/abafa2}

  \item Eight co-authored publications
    (\href{https://ui.adsabs.harvard.edu/search/filter_doctype_facet_hier_fq_doctype=NOT&filter_doctype_facet_hier_fq_doctype=*%3A*&filter_doctype_facet_hier_fq_doctype=doctype_facet_hier%3A%221%2FNon-Article%2FProposal%22&fq=%7B!type%3Daqp%20v%3D%24fq_database%7D&fq=%7B!type%3Daqp%20v%3D%24fq_doctype%7D&fq_database=(database%3Aastronomy%20OR%20database%3Aphysics)&fq_doctype=(*%3A*%20NOT%20doctype_facet_hier%3A%221%2FNon-Article%2FProposal%22)&p_=0&q=((%20author%3A%22burrow%2C%20anthony%22)%20AND%20year%3A2017-2024)&sort=date%20desc%2C%20bibcode%20desc}
          {view NASA ADS query})
    of scientific and data analyses with collaborators.


\resumeSubHeadingListEnd


%
% Presentations
%


\section{Presentations}
\resumeSubHeadingListStart

  \resumeItem{POISE Collaboration Meeting, August 2023}
    {\textit{Extrapolation of Type Ia Spectra into the Near-Infrared Using
     PCA}; a final discussion leading to \cite{Burrow_etal_2024}.}
  \resumeItem{POISE Collaboration Meeting, July 2022}
    {\textit{Extrapolation of Type Ia Spectra into the Near-Infrared Using
     PCA}; the beginning of the project leading to \cite{Burrow_etal_2024}.}
  \resumeItem{CSP Collaboration Workshop, September 2020}
    {\textit{Carnegie Supernova Project: Classification of Type Ia
     Supernovae}; a presentation of the publication by
     \cite{Burrow_etal_2020}.}
  \resumeItem{American Astronomical Society Winter Conference, January 2017}
    {Poster presentation highlighting my undergraduate research.}

\resumeSubHeadingListEnd


%
% Awards
%


\section{Awards}
\resumeSubHeadingListStart

\resumeItem{Avenir Foundation Graduate Student Fellowship}{}

  Spring 2022, Summer 2022; \textit{University of Oklahoma}

\resumeItem{Provost’s Certificate of Distinction in Teaching}{}

  Fall 2019; \textit{University of Oklahoma}

\resumeItem{Award for Outstanding Scholarship by a Graduating Senior}{}

  May 2017; \textit{University of Oklahoma Homer L. Dodge Department of Physics and Astronomy}

\resumeItem{William Schriever Award for Outstanding Scholarship in Physics \& Astronomy}{}

  2014-2015; \textit{University of Oklahoma}

\resumeSubHeadingListEnd


%
% Additional Work Experience
%


\section{Additional Work Experience}
\resumeSubHeadingListStart
  \setlength\itemsep{-2pt}

  \resumeSubheading
    {Graduate Teaching Assistant}{Aug. 2019 -- Dec. 2021}
    {University of Oklahoma}{Norman, OK}

    \begin{itemize}
    \setlength\itemsep{0pt}
    % \item Provided lectures and guide group discussions on topics in-class to
    %   undergraduate students in astronomy and physics courses.
    % \item Led students with hands-on operation of telescopes during
    %   astronomy labs.
    % \item Graded and evaluated students' work for higher-level undergraduate
    %   astronomy courses.
    \item Delivered lectures and facilitated in-class group discussions in
     undergraduate astronomy and physics courses, focusing on foundational
     concepts and problem-solving strategies.
    \item Led hands-on astronomy lab sessions, instructing students in the
     operation of telescopes.
    \item Heavily supported an undergraduate Introduction to Research course,
     emphasizing hands-on research practices including data analysis, coding in
     Python, and literature review.
    \item Graded assignments and provided individualized feedback for advanced
     undergraduate astronomy students.
    \end{itemize}

\resumeSubHeadingListEnd


\end{document}
