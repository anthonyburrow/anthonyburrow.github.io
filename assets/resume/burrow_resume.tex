\documentclass[letterpaper,11pt]{article}

\usepackage{latexsym}
\usepackage[empty]{fullpage}
\usepackage{titlesec}
\usepackage{marvosym}
\usepackage[usenames,dvipsnames]{color}
\usepackage{verbatim}
\usepackage{enumitem}
\usepackage{xifthen}
% \usepackage[pdftex]{hyperref}
\usepackage{fancyhdr}
\usepackage{fontawesome}
\usepackage{multirow}
\usepackage{multicol}
\usepackage{natbib}
\usepackage{bibentry}
\bibliographystyle{apalike}


\usepackage{my_macros}

\definecolor{xlinkcolor}{cmyk}{1,1,0,0}
\usepackage[
  colorlinks=true,
  linkcolor=xlinkcolor,
  citecolor=xlinkcolor,
  filecolor=xlinkcolor,
  urlcolor=xlinkcolor,
]{hyperref}

\pagestyle{fancy}
\fancyhf{} % clear all header and footer fields
\fancyfoot{}
\renewcommand{\headrulewidth}{0pt}
\renewcommand{\footrulewidth}{0pt}

% Adjust margins
\addtolength{\oddsidemargin}{-0.375in}
\addtolength{\evensidemargin}{-0.375in}
\addtolength{\textwidth}{1in}
\addtolength{\topmargin}{-.5in}
\addtolength{\textheight}{1.0in}

\urlstyle{same}

\raggedbottom
\raggedright
\setlength{\tabcolsep}{0in}

% Sections formatting
\titleformat{\section}{
  \vspace{-4pt}\scshape\raggedright\large
}{}{0em}{}[\color{black}\titlerule \vspace{-5pt}]

%-------------------------
% Custom commands
\newlength{\mylen}
\setbox1=\hbox{$\bullet$}\setbox2=\hbox{\tiny$\bullet$}
\setlength{\mylen}{\dimexpr0.5\ht1-0.5\ht2}
\renewcommand\labelitemi{\raisebox{\mylen}{\small$\bullet$}}

\newcommand{\resumeItem}[2]{
  \item\small{
    \textbf{#1}{: #2 \vspace{-2pt}}
  }
}

\newcommand{\smallHeading}[1]{\small{\textbf{#1}:\vspace{-5pt}}}

\newcommand{\resumeSubheading}[4]{
  \vspace{-1pt}\item
    \begin{tabular*}{0.97\textwidth}{l@{\extracolsep{\fill}}r}
      \textbf{#1} & #2 \\
      \textit{\small#3} & \textit{\small #4} \\
    \end{tabular*}\vspace{-5pt}
}

\newcommand{\resumeSubItem}[2]{\resumeItem{#1}{#2}\vspace{-4pt}}

\renewcommand{\labelitemii}{$\circ$}

\newcommand{\resumeSubHeadingListStart}{\begin{itemize}[leftmargin=*]}
\newcommand{\resumeSubHeadingListEnd}{\end{itemize}}
\newcommand{\resumeItemListStart}{\begin{itemize}}
\newcommand{\resumeItemListEnd}{\end{itemize}\vspace{-5pt}}

\newcommand{\paperEntry}[5]{
  \item #1, ..., \textbf{Burrow, Anthony}, et al. (#2). \textit{#3}. {#4}
  \ifthenelse{\isempty{#5}}{}{, doi: \href{https://doi.org/#5}{#5}}
}

\newcommand{\paperEntryFirstAuthor}[5]{
  \item \textbf{Burrow, Anthony}, #1, et al. (#2). \textit{#3}. #4
  \ifthenelse{\isempty{#5}}{}{, doi: \href{https://doi.org/#5}{#5}}
}

\newcommand{\educationEntry}[5]{
  \vspace{-1pt}\item
    \begin{tabular*}{0.97\textwidth}{l@{\extracolsep{\fill}}r}
      \textbf{#1}; \textit{GPA: #2}  & #3 \\
      \textit{\small#4} & \textit{\small #5} \\
    \end{tabular*}\vspace{-5pt}
}

\newcommand{\emphasize}[1]{\textbf{#1}}

% -------------------------------------------

\begin{document}

\nobibliography{mybib}


% Header
\begin{tabular*}{\textwidth}{l@{\extracolsep{\fill}}l}
  \multirow{3}{*}{\textbf{\huge Anthony Burrow}}
  & \faEnvelope \enspace anthony.r.burrow@gmail.com \\
  & \faGithub \enspace \href{http://www.github.com/anthonyburrow}{github.com/anthonyburrow} \\
  & \faLinkedin \enspace \href{https://www.linkedin.com/in/anthony-burrow}{linkedin.com/in/anthony-burrow} \\
\end{tabular*}


\section{Education}
\resumeSubHeadingListStart

  \educationEntry
    {Ph.D. in Physics}{3.92}{Aug. 2019 -- July 2024}
    {University of Oklahoma; \href{https://shareok.org/handle/11244/340468}{Dissertation on SHAREOK}}{Norman, OK}
  \educationEntry
    {B.S. in Astrophysics}{3.91}{Aug. 2014 -- May. 2017}
    {University of Oklahoma}{Norman, OK}

\resumeSubHeadingListEnd


\section{Technical Skills}

  \begin{tabular*}{0.97\textwidth}{l@{\extracolsep{\fill}}l@{\extracolsep{\fill}}l}
    \vspace{3pt}
    \textbf{Programming} & \textbf{Platforms} & \textbf{Technologies} \\
    Python, C/C++, C\#, SQL, Bash, & Linux/UNIX, Windows & Git (Version Control), \LaTeX, RStudio, \\
    Fortran, IDL, Make/Makefiles, CMake & & Microsoft Office, Mathematica, IRAF \\
    \vspace{-3pt}
  \end{tabular*}

  \vspace{-3pt}
  \textbf{Experience with Python Libraries}

  \vspace{1pt}
  \begin{tabular*}{0.97\textwidth}{l@{\extracolsep{\fill}}l@{\extracolsep{\fill}}l@{\extracolsep{\fill}}l@{\extracolsep{\fill}}l@{\extracolsep{\fill}}l}
    % \vspace{3pt}
    NumPy & SciPy & scikit-learn & Astropy & Tensorflow & pandas \\
    matplotlib & Jupyter & Cython & pybind11 & GPy & george \\
    % \vspace{-3pt}
  \end{tabular*}

  \vspace{6pt}
  \textbf{Strengths}
  \vspace{1pt}

  Data Analysis \hfill$\circ$\hfill
  Data Visualization \hfill$\circ$\hfill
  Cluster Analysis \hfill$\circ$\hfill
  Classification \hfill$\circ$\hfill
  Hierarchical Bayesian Modeling \\
  Statistics \hfill$\circ$\hfill
  Machine Learning \hfill$\circ$\hfill
  Interpolation \& Extrapolation \hfill$\circ$\hfill
  Numerical Computation \\
  Software Development \hfill$\circ$\hfill
  Unit Testing \hfill$\circ$\hfill
  Debugging \hfill$\circ$\hfill
  Scripting \hfill$\circ$\hfill
  Automation \hfill$\circ$\hfill
  Optimization


\section{Research Experience}
\resumeSubHeadingListStart

  \normalsize
  \resumeSubheading
    {Graduate Research Assistant}{July 2019 -- Present}
    {University of Oklahoma, Advised by Dr. Eddie Baron}{Norman, OK}

    \vspace{5pt}
    \hspace*{0.85em}
    \begin{minipage}{0.95\textwidth}
    My research focuses on a statistical treatment of observations of Type Ia
    supernovae (SNe~Ia). As a result, I identified correlations between many
    spectroscopic and photometric properties of these supernovae, which will
    lead to the enhancement of supernova models all around the scientific
    community.
    \end{minipage}
    \begin{itemize}\small
      \item \emphasize{Develop Python software} to implement several
        \emphasize{machine-learning} techniques to \emphasize{model} the
        behavior of SNe~Ia.
      \item \emphasize{Statistically analyze data}, resulting in two
        first-author \emphasize{publications} and a dissertation that
        illustrate the effectiveness of my results in performing
        \emphasize{classifications} and \emphasize{predictions}.
      \item Work in conjunction with the Precision Observations of Infant
        Supernova Explosions (POISE), a larger \emphasize{collaboration}
        between several other universities and facilities around the world.
      \item Utilize the \texttt{PHOENIX} radiative transfer code in a
        \emphasize{supercomputing} environment with \emphasize{Slurm} workload
        management scripts to generate synthetic spectrum \emphasize{models} to
        better understand the diversity of SNe~Ia.
    \end{itemize}
    \hspace*{1.1em}\smallHeading{Products}
    \begin{itemize}\small
      \item \href{https://github.com/anthonyburrow/SNEx}{SNEx} (Python):
        Spectrum \emphasize{extrapolation} into the near-infrared using PCA
        \citep{Burrow_etal_2024}.
      \item \href{https://github.com/anthonyburrow/spextractor}{Spextractor} (Python):
        \emphasize{Fast spectrum-smoothing} using Gaussian processes; spectrum
        \emphasize{preprocessing}; other useful features.
      \item \href{https://github.com/anthonyburrow/SNIaDCA}{SNIaDCA} (Python):
        Wrapper for \emphasize{classifying} SNe~Ia with Gaussian mixture models
        \citep{Burrow_etal_2020}.
    \end{itemize}\vspace{-5pt}

  \normalsize
  \resumeSubheading
    {Undergraduate Research Assistant}{June 2015 -- May 2017}
    {University of Oklahoma, Advised by Dr. John Wisniewski}{Norman, OK}

    \vspace{5pt}
    \hspace*{0.85em}
    \begin{minipage}{0.95\textwidth}
    My work concentrated on the observation, reduction, and analysis of
    data taken of star clusters to improve our understanding of stars with
    variable circumstellar disks.
    \end{minipage}
    \begin{itemize}\small
      \item \emphasize{Reduce} observed data by removing multiple sources of
        noise from raw \texttt{FITS} images of star clusters using
        \texttt{IRAF}.
      \item \emphasize{Model} the light profile of stars on images to calculate
        their PSF photometry using \texttt{IRAF}.
      \item Create Python and IDL scripts needed to \emphasize{analyze data}
        and propagate errors derived from observations.
      \item Conduct multiple remote \emphasize{observations} using a 0.5m
        telescope at the Apache Point Observatory
    \end{itemize}

\resumeSubHeadingListEnd


\section{Additional Projects}
\resumeSubHeadingListStart

  \item \href{https://github.com/anthonyburrow/Rad1D}{Rad1D}:
    A 1D radiative transfer code written in C++ featuring a Python wrapper for
    easy implementation into Python for analysis and plotting. This program
    converges a solution to wavelength-dependent radiative transfer equations,
    which describes how light behaves as it passes through a medium as a
    function of optical depth.
  \item \href{https://github.com/anthonyburrow/Hydro1D}{Hydro1D}:
    A 1D hydrodynamical code written primarily in C++ with some Python
    additions. This program models the fluid dynamics of a massive (10
    M$_\odot$) star undergoing a collapse and a shock event, which leads to a
    core-collapse supernova.

\resumeSubHeadingListEnd


\section{Publications}

  \smallHeading{First-Authored Papers}

  \resumeItemListStart
    % \item Burrow et al. (\textit{ApJ, in press})
    % \item \bibentry{Burrow_etal_2020}
    \paperEntryFirstAuthor{Baron, E., Burns, Christopher R.}{2024}
      {Extrapolation of Type Ia Supernova Spectra into the Near-Infrared Using PCA}
      {arXiv e-prints}{10.48550/arXiv.2404.04724}
    \paperEntryFirstAuthor{Baron, E., Ashall, C.}{2020}
      {Carnegie Supernova Project: Classification of Type Ia Supernovae}
      {\apj}{10.3847/1538-4357/abafa2}
  \resumeItemListEnd

  \smallHeading{Relevant Co-Authored Papers}

  \resumeItemListStart
    % \paperEntry{Shahbandeh, Melissa}{2024}
    %            {Unraveling cosmic dust origins: JWST revelations from legacy observations of SN 2023dbc}
    %            {JWST Proposal. Cycle 3, ID. \#6213}{}
    % \paperEntry{Shahbandeh, Melissa}{2024}
    %            {Probing Early Dust Formation in the Universe via Stripped-Envelope Supernovae}
    %            {JWST Proposal. Cycle 3, ID. \#6583}{}
    % \paperEntry{DerKacy, James M.}{2024}
    %            {Examining the Heart of Type Ia Supernova 2021aefx with Ultra-Late Time Spectra}
    %            {JWST Proposal. Cycle 3, ID. \#6582}{}
    % \paperEntry{DerKacy, James M.}{2024}
    %            {The Full Picture: Determining the Ultra-Late Time MIR Flux Redistribution in SN 2021aefx}
    %            {JWST Proposal. Cycle 3, ID. \#6023}{}
    % \paperEntry{Ashall, Chris}{2024}
    %            {Late Time Spectroscopy of Type Ia Supernovae: Determining the Explosion Mechanism and Elemental Production}
    %            {JWST Proposal. Cycle 3, ID. \#5057}{}
    \paperEntry{DerKacy, James M.}{2024}
               {JWST MIRI/Medium Resolution Spectrograph (MRS) Observations and Spectral Models of the Underluminous Type Ia Supernova 2022xkq}
               {\apj}{10.3847/1538-4357/ad0b7b}
    \paperEntry{Shahbandeh, Melissa}{2024}
               {JWST NIRSpec+MIRI Observations of the nearby Type IIP supernova 2022acko}
               {arXiv e-prints}{10.48550/arXiv.2401.14474}
    % \paperEntry{DerKacy, James M.}{2023}
    %            {The UV Future is Now: Tapping Hubble's UV Spectral Archive to Drive Current and Future Type Ia Supernova Science}
    %            {HST Proposal. Cycle 31, ID. \#17555}{}
    % \paperEntry{Shahbandeh, Melissa}{2023}
    %            {Near- and Mid-IR Observations to Probe Dust Formation in the Remarkably Nearby Stripped-Envelope Supernova 2023dbc}
    %            {JWST Proposal. Cycle 2, ID. \#4520}{}
    % \paperEntry{Shahbandeh, Melissa}{2023}
    %            {Probing Early Dust Formation in the Universe via Stripped-Envelope Supernovae}
    %            {JWST Proposal. Cycle 2, ID. \#4217}{}
    % \paperEntry{DerKacy, James M.}{2023}
    %            {Examining the Heart of Type Ia Supernova 2021aefx with Ultra-Late Time Spectra}
    %            {JWST Proposal. Cycle 2, ID. \#3726}{}
    \paperEntry{Yarbrough, Zach}{2023}
               {Direct analysis of the broad-line SN 2019ein: connection with the core-normal SN 2011fe}
               {\mnras}{10.1093/mnras/stad758}
    % \paperEntry{Shahbandeh, Melissa}{2023}
    %            {Near- and Mid-IR Observations to Probe Dust Formation in the Remarkably Nearby Stripped-Envelope Supernova 2023dbc}
    %            {JWST Proposal. Cycle 1, ID. \#4436}{}
    \paperEntry{DerKacy, J. M.}{2023}
               {JWST Low-resolution MIRI Spectral Observations of SN 2021aefx: High-density Burning in a Type Ia Supernova}
               {\apj}{10.3847/2041-8213/acb8a8}
    % \paperEntry{Lu, J.}{2022}
    %            {POISE Transient Classification Report for 2022-03-18}
    %            {Transient Name Server Classification Report}{}
    % \paperEntry{Ashall, Chris}{2021}
    %            {MIR Spectroscopy of Type Ia Supernovae: The Key to Unlocking their Explosions and Element Production}
    %            {JWST Proposal. Cycle 1, ID. \#2114}{}
    \paperEntry{Burns, C.}{2021}
               {Introducing POISE: Precision Observations of Infant Supernova Explosions}
               {\atel}{}
  \resumeItemListEnd


\section{Technical Presentations}
\resumeSubHeadingListStart

  \resumeItem{POISE Collaboration Meeting, August 2023}
    {\textit{Extrapolation of Type Ia Spectra into the Near-Infrared Using
     PCA}; a final discussion leading to \cite{Burrow_etal_2024}.}
  \resumeItem{POISE Collaboration Meeting, July 2022}
    {\textit{Extrapolation of Type Ia Spectra into the Near-Infrared Using
     PCA}; the beginning of the project leading to \cite{Burrow_etal_2024}.}
  \resumeItem{CSP Collaboration Workshop, September 2020}
    {\textit{Carnegie Supernova Project: Classification of Type Ia
     Supernovae}; a presentation of the publication by
     \cite{Burrow_etal_2020}.}
  \resumeItem{American Astronomical Society Winter Conference, January 2017}
    {Poster presentation highlighting my undergraduate research.}
  \resumeItem{OU REU Program, Summer 2015}
    {Several presentations describing the results of my undergraduate
    research during this program.}

\resumeSubHeadingListEnd


\section{Teaching Experience}
\resumeSubHeadingListStart

  \resumeSubheading
    {Graduate Teaching Assistant}{Aug. 2019 -- Dec. 2021}
    {University of Oklahoma}{Norman, OK}

    \begin{itemize}
    \item Provide lectures and guide group discussions on topics in-class to
      undergraduate students in Introductory Astronomy and Physics courses.
    \item Lead students with hands-on operation of telescopes during
      astronomy labs.
    \item Grading and evalution for higher-level undergraduate astronomy
      ourses, such as Galaxies \& Cosmology and Stellar Astrophysics.
    \end{itemize}

\resumeSubHeadingListEnd


\pagebreak


\section{Relevant Coursework}
\resumeSubHeadingListStart

  \resumeItem{Machine Learning}
    {(See \href{https://github.com/anthonyburrow/MachineLearning}{repository}.)}

    Discusses advanced statistical techniques and machine learning concepts such as:

    \vspace{-10pt}
    \begin{multicols}{3}
    \begin{itemize}[nolistsep]
        \item Regression Analysis
        \item Gaussian Processes
        \item Cluster Analysis
        \item Probabilistic Classification
        \item Kernel Density Estimation
        \item Neural Networks
    \end{itemize}
    \end{multicols}
    \vspace{-12pt}

  \resumeItem{Numerical Methods}
    {(See \href{https://github.com/anthonyburrow/NumericalMethods}{repository}.)}

    Discusses the most common problems that arise in computation and methods
    to address them, such as root-solving, solving systems of equations, and
    numerically solving ordinary and partial differential equations. This
    course also provides an introduction to high-performance computing,
    describing the architecture of modern supercomputing and giving practice
    with parallelization of computation using interfaces such as MPI and
    OpenMP.

  \resumeItem{Stellar Atmospheres}{}

    Largely focuses on understanding the physics of light propagating
    through a medium by solving mathematically complex systems of equations
    that describe radiative transfer, which are only solvable numerically in
    application. This is a fundamental approach to understanding and modeling
    spectra observed from astronomical objects.

  \resumeItem{Core \& Advanced Physics Courses}{}

    An assortment of courses that allow for the fundamental understanding of
    physics at the post-graduate level, including Classical Mechanics,
    Statistical Mechanics, Quantum Mechanics, and Electrodynamics, and
    classes covering advanced topics in physics, including Quantum Mechanics
    of Atoms and General Relativity.

\resumeSubHeadingListEnd



% TODO:
% - Link to dissertation
% - clean up readmes that are linked
% - update linkedin
% - awards section
% - Lunar sooners?


\end{document}
