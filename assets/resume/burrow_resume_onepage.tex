\documentclass[letterpaper,11pt]{article}

\usepackage{latexsym}
\usepackage[empty]{fullpage}
\usepackage{titlesec}
\usepackage{marvosym}
\usepackage[usenames,dvipsnames]{color}
\usepackage{verbatim}
\usepackage{enumitem}
\usepackage{xifthen}
% \usepackage[pdftex]{hyperref}
\usepackage{fancyhdr}
\usepackage{fontawesome}
\usepackage{multirow}
\usepackage{multicol}
\usepackage{natbib}
\usepackage{bibentry}
\bibliographystyle{apalike}


\usepackage{my_macros}

\definecolor{xlinkcolor}{cmyk}{1,1,0,0}
\usepackage[
  colorlinks=true,
  linkcolor=xlinkcolor,
  citecolor=xlinkcolor,
  filecolor=xlinkcolor,
  urlcolor=xlinkcolor,
]{hyperref}

\pagestyle{fancy}
\fancyhf{} % clear all header and footer fields
\fancyfoot{}
\renewcommand{\headrulewidth}{0pt}
\renewcommand{\footrulewidth}{0pt}

% Adjust margins
\addtolength{\oddsidemargin}{-0.375in}
\addtolength{\evensidemargin}{-0.375in}
\addtolength{\textwidth}{1in}
\addtolength{\topmargin}{-.5in}
\addtolength{\textheight}{1.0in}

\urlstyle{same}

\raggedbottom
\raggedright
\setlength{\tabcolsep}{0in}

% Sections formatting
\titleformat{\section}{
  \vspace{-4pt}\scshape\raggedright\large
}{}{0em}{}[\color{black}\titlerule \vspace{-5pt}]

%-------------------------
% Custom commands
\newlength{\mylen}
\setbox1=\hbox{$\bullet$}\setbox2=\hbox{\tiny$\bullet$}
\setlength{\mylen}{\dimexpr0.5\ht1-0.5\ht2}
\renewcommand\labelitemi{\raisebox{\mylen}{\small$\bullet$}}

\newcommand{\resumeItem}[2]{
  \item\small{
    \textbf{#1}{: #2 \vspace{-2pt}}
  }
}

\newcommand{\smallHeading}[1]{\small{\textbf{#1}:\vspace{-5pt}}}

\newcommand{\resumeSubheading}[4]{
  \vspace{-1pt}\item
    \begin{tabular*}{0.97\textwidth}{l@{\extracolsep{\fill}}r}
      \textbf{#1} & #2 \\
      \textit{\small#3} & \textit{\small #4} \\
    \end{tabular*}\vspace{-5pt}
}

\newcommand{\resumeSubItem}[2]{\resumeItem{#1}{#2}\vspace{-4pt}}

\renewcommand{\labelitemii}{$\circ$}

\newcommand{\resumeSubHeadingListStart}{\begin{itemize}[leftmargin=*]}
\newcommand{\resumeSubHeadingListEnd}{\end{itemize}}
\newcommand{\resumeItemListStart}{\begin{itemize}}
\newcommand{\resumeItemListEnd}{\end{itemize}\vspace{-5pt}}

\newcommand{\paperEntry}[5]{
  \item #1, ..., \textbf{Burrow, Anthony}, et al. (#2). \textit{#3}. {#4}
  \ifthenelse{\isempty{#5}}{}{, doi: \href{https://doi.org/#5}{#5}}
}

\newcommand{\paperEntryFirstAuthor}[5]{
  \item \textbf{Burrow, Anthony}, #1, et al. (#2). \textit{#3}. #4
  \ifthenelse{\isempty{#5}}{}{, doi: \href{https://doi.org/#5}{#5}}
}

\newcommand{\paperEntryShort}[4]{
  \href{https://doi.org/#1}{\textbf{Burrow, Anthony}, et al. (#2)}.
  \textit{#3}. #4
}

% \newcommand{\educationEntry}[5]{
%   \vspace{-1pt}\item
%     \begin{tabular*}{0.97\textwidth}{l@{\extracolsep{\fill}}r}
%       \textbf{#1}\ifthenelse{\isempty{#2}}{}{; GPA: #2}  & #3 \\
%       \textit{\small#4} & \textit{\small #5} \\
%     \end{tabular*}\vspace{-5pt}
% }

\newcommand{\educationEntry}[5]{
  \vspace{-1pt}
  \item
    \begin{tabular*}{0.97\textwidth}[t]{l}
      \textbf{#1}\ifthenelse{\isempty{#4}}{}{ (#4)} \\
      #3, \textit{\small#2}
      \ifthenelse{\isempty{#5}}{}{\\ #5 \\}
    \end{tabular*}
  \vspace{-5pt}
}

% \newcommand{\emphasize}[1]{\textbf{#1}}
\newcommand{\emphasize}[1]{#1}

% -------------------------------------------

\begin{document}

\nobibliography{mybib}


% Header
\begin{tabular*}{\textwidth}{l@{\extracolsep{\fill}}l}
  \multirow{3}{*}{\textbf{\huge Anthony Burrow, Ph.D.}}
  & \faEnvelope \enspace anthony.r.burrow@gmail.com \\
  & \faLinkedin \enspace \href{https://www.linkedin.com/in/anthony-burrow}{linkedin.com/in/anthony-burrow} \\
  & \faHome \enspace \href{https://anthonyburrow.github.io}{anthonyburrow.github.io} \\
  % & \faGithub \enspace \href{http://www.github.com/anthonyburrow}{github.com/anthonyburrow} \\
\end{tabular*}


\vspace{10pt}
\begin{minipage}[t]{0.48\textwidth}
\section{Summary}

Research scientist with a Ph.D. in Physics and a robust background in applying
machine-learning concepts to complex datasets in astrophysics. Extensive work
utilizing Python, C/C++, and more to develop software tools which have shown to
be significant contributions to the scientific community. Experienced with the
entire data science life cycle: identifying problems, data wrangling, and model
deployment, evaluation, and maintenance.

\end{minipage}%
\hspace{0.04\textwidth}%
\begin{minipage}[t]{0.48\textwidth}
\section{Education}

\resumeSubHeadingListStart

  \vspace{-5pt}
  \setlength\itemsep{2pt}
  \educationEntry
    {Ph.D. in Physics}
    {University of Oklahoma}{July 2024}{GPA: 3.92}
    {\href{https://shareok.org/handle/11244/340468}{Dissertation on SHAREOK}}
  \educationEntry
    {Professional Certificate, Machine Learning}
    {IBM}{\textit{Ongoing}}{}
    {}
  \educationEntry
    {Professional Certificate, Data Science}
    {IBM}{Aug. 2024}{\href{https://www.coursera.org/account/accomplishments/specialization/certificate/1CX1HP3UEHJ1}{View}}
    {}
  \educationEntry
    {B.S. in Astrophysics}
    {University of Oklahoma}{May 2017}{GPA: 3.91}
    {}

\resumeSubHeadingListEnd

\end{minipage}


\section{Technical Skills}

  \begin{tabular*}{0.97\textwidth}{l@{\extracolsep{\fill}}l@{\extracolsep{\fill}}l}
    \vspace{3pt}
    \textbf{Programming}: & \textbf{Platforms}: & \textbf{Technologies}: \\
    Python, SQL, C/C++, C\#, Bash & Linux/UNIX, Windows & Git, JupyterLab, RStudio, \LaTeX, Slurm \\
    \vspace{-3pt}
  \end{tabular*}

  \vspace{-3pt}
  \textbf{Experience with Python Libraries}:

  NumPy \hfill
  pandas \hfill
  scikit-learn \hfill
  matplotlib \hfill
  SciPy \hfill
  Astropy \hfill
  Tensorflow \hfill
  GPy

  \vspace{6pt}
  \textbf{Data Science Skills}:
  \vspace{1pt}

  Machine Learning \hfill
  Data Wrangling \hfill
  Statistics \hfill
  Data Analysis \hfill
  Data Visualization \hfill
  Model Evaluation

  Regression \hfill
  Classification \hfill
  Parameter Optimization \hfill
  Cluster Analysis \hfill
  Dimensionality Reduction

\section{Research Experience}
\resumeSubHeadingListStart

  \normalsize
  \resumeSubheading
    {Graduate Research Assistant}{July 2019 -- Present}
    {University of Oklahoma, Advised by Dr. Eddie Baron}{Norman, OK}

    \begin{itemize}\small
      \item \emphasize{Develop Python software} to implement
        \emphasize{machine-learning} techniques to \emphasize{model} the
        behavior of supernovae.
      \item Perform thorough \emphasize{preprocessing},
        \emphasize{standardization}, and \emphasize{feature engineering} of
        spectroscopic data.
      \item Conduct detailed \emphasize{statistical analyses}, resulting in two
        \emphasize{publications} in a peer-reviewed journal (ApJ).
      \item \emphasize{Collaborate} with leading researchers from several other
        universities and facilities around the world (CSP, POISE).
      \item \emphasize{Present} results to peers and collaborators at meetings
        and conferences.
      \item \emphasize{Synthesize models} in a \emphasize{supercomputing}
        environment with \emphasize{Slurm} scripts using \texttt{PHOENIX}
        radiative transfer code.
    \end{itemize}
    \hspace*{1.1em}\smallHeading{Products}
    \begin{itemize}\small
      \item \paperEntryShort{10.3847/1538-4357/ad3c45}{2024}
            {Extrapolation of Type Ia Supernova Spectra into the NIR Using PCA}
            {\apj}
      \item \paperEntryShort{10.3847/1538-4357/abafa2}{2020}
            {Carnegie Supernova Project: Classification of Type Ia Supernovae}
            {\apj}
      \item \href{https://github.com/anthonyburrow/SNEx}{SNEx} (Python):
        Spectrum \emphasize{extrapolation} into near-infrared wavelengths using
        \emphasize{principal component analysis}.
      \item \href{https://github.com/anthonyburrow/spextractor}{Spextractor} (Python):
        Fast spectrum-smoothing using \emphasize{Gaussian process regression}.
      \item \href{https://github.com/anthonyburrow/SNIaDCA}{SNIaDCA} (Python):
        Wrapper for probablistically \emphasize{classifying} supernovae with
        \emphasize{Gaussian mixture models}.
    \end{itemize}\vspace{-5pt}

  \normalsize
  \resumeSubheading
    {Undergraduate Research Assistant}{June 2015 -- May 2017}
    {University of Oklahoma, Advised by Dr. John Wisniewski}{Norman, OK}

    \begin{itemize}\small
      \item \emphasize{Calibrated} observed data by removing multiple sources
        of noise from raw \texttt{FITS} images of stars using \texttt{IRAF}.
      \item \emphasize{Modeled} the observed light profile of stars on
        images using \texttt{IRAF} to calculate their brightness values.
      \item Created Python and IDL scripts needed to \emphasize{analyze data}
        and propagate errors derived from observations.
      \item Conducted multiple remote \emphasize{observations} at the Apache
        Point Observatory to obtain more raw data for analysis.
      \item \emphasize{Presented results} at the American Astronomical Society
        conference.
    \end{itemize}

\resumeSubHeadingListEnd


\end{document}
